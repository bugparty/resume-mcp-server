\documentclass[11pt, a4paper]{awesome-cv}

% Configure page margins with geometry
\geometry{left=1.4cm, top=.8cm, right=1.4cm, bottom=1.8cm, footskip=.5cm}
\fontdir[fonts/]
\colorlet{awesome}{awesome-red}

% Set false if you don't want to highlight section with awesome color
\setbool{acvSectionColorHighlight}{true}

% If you would like to change the social information separator from a pipe (|) to something else
\renewcommand{\acvHeaderSocialSep}{\quad\textbar\quad}

%-------------------------------------------------------------------------------
%       CJK (Chinese/Japanese/Korean) Font Support
%-------------------------------------------------------------------------------
% Add Chinese font support using xeCJK
\usepackage{xeCJK}
% Set Chinese fonts (fallback to available system fonts)
\setCJKmainfont{Droid Sans Fallback}
\setCJKsansfont{Droid Sans Fallback}
\setCJKmonofont{Droid Sans Fallback}

%-------------------------------------------------------------------------------
%	PERSONAL INFORMATION
%	Comment any of the lines below if they are not required
%-------------------------------------------------------------------------------
% Available options: circle|rectangle,edge/noedge,left/right
\name{徐**}{-}
\position{大模型算法}

%-------------------------------------------------------------------------------
\begin{document}

% Print the header with above personal informations
% Give optional argument to change alignment(C: center, L: left, R: right)
\makecvheader[C]

% Print the footer with 3 arguments(<left>, <center>, <right>)
% Leave any of these blank if they are not needed
\makecvfooter
  {\today}
  {徐** -~~~\textperiodcentered~~~Résumé}
  {\thepage}

%-------------------------------------------------------------------------------
%	CV/RESUME CONTENT
%	Each section is imported separately, open each file in turn to modify content
%-------------------------------------------------------------------------------
\cvsection{Summary}
\begin{cvitems}  \item {聚焦于dev infra 领域智能化。 熟悉大模型应用,agent, aiOps ,数据库平台}  \item {期望职位: 大模型算法 @ 南京, 70-100K}  \item {期望职位: 大模型算法 @ 上海, 90-120K}  \item {期望职位: 大模型算法 @ 杭州, 80-110K}\end{cvitems}


\cvsection{Professional Experience}
\begin{cventries}  \cventry
    {高级算法专家}
  {北京字节跳动科技有限公司}
    {None}
    {2024.04 - 至今}
    {      \begin{cvitems}        \item {LLM应用,agent,豆包}      \end{cvitems}    }  \cventry
    {算法专家}
  {北京三快在线科技有限公司}
    {None}
    {2021.08 - 至今}
    {      \begin{cvitems}        \item {数据库自治,LLM 应用}      \end{cvitems}    }  \cventry
    {算法工程师}
  {华为}
    {None}
    {2017.04 - 2021.08}
    {      \begin{cvitems}        \item {从事 aiOps 、广告、用户体验相关算法的研究和落地,包括:}        \item {指标异常检测:使用时间序列处理、IForest 、概率分布建模等算法,基于时序特征,构建异常检测能力,支撑现网数百万 KPI 分钟级实时异常检测,为华为消费者云服务100+业务提供质量保障。}        \item {广告人群扩展:基于 GBDT 进行特征筛选,使用 LR/Linear SVM 进行分类建模,对广告主选定人群进行相似人群扩展,提高投放转化效果,1:1扩散转化率提升200\%+。}        \item {广告故障定位:基于广告业务流程和指标关联,使用 MDRCA 、对比分析、变更关联等算子,构建广告业务智能化故障定位能力,支持流程图注入及在线学习专家经验。这段准确率70\%+,部署于2个业务。}        \item {舆情监控:使用 BERT 算法,对 APP 反馈、电话工单、互联网等来源的用户舆情数据做 NLP 分类、标签化。基于标签化数据进行实时检测,发现舆情问题;挖掘热点话题,追踪舆情问题;覆盖10+APP。}        \item {灰度评估:使用分布建模、假设检验方法,对灰度上线过程中的用户体验、核心接口成功率、机器负载等指标进行灰度-现网对比评分,并对问题指标进行下钻分析,给出根因。已接入云服务10+业务。}        \item {用户体验分析:根据用户调查问卷,分析各业务指标与 NPS 的关系及权重,对产品改进提供方向及优先级建议,部署于云服务2个业务。}      \end{cvitems}    }  \cventry
    {算法工程师}
  {中国银联技术开发中心}
    {None}
    {2016.07 - 2016.10}
    {      \begin{cvitems}        \item {实习经历:在银联技术清算室实习,团队负责利用机器学习从历史清算数据预测未来清算交易额,用来对未}        \item {来运营、清算配置规划提供数据支持。项目使用了 SVM、LSTM 算法。此外还负责部分清算交易}        \item {数据分析、统计报表工作。}      \end{cvitems}    }\end{cventries}


\cvsection{Education}
\begin{cventries}  \cventry
    {博士}
  {华东师范大学}
    {None}
    {2023 - 2027}
    {      \begin{cvitems}        \item {专业: 电子与信息技术}      \end{cvitems}    }  \cventry
    {硕士}
  {南京航空航天大学}
    {None}
    {2014 - 2017}
    {      \begin{cvitems}        \item {专业: 电子与通信工程}        \item {国家奖学金}        \item {全国研究生数学建模竞赛一等奖}      \end{cvitems}    }  \cventry
    {本科}
  {南京航空航天大学}
    {None}
    {2010 - 2014}
    {      \begin{cvitems}        \item {专业: 信息工程}        \item {美国大学生数学建模竞赛(MCM)一等奖}      \end{cvitems}    }\end{cventries}


%-------------------------------------------------------------------------------
\end{document}

