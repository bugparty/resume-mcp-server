\documentclass[11pt, a4paper]{awesome-cv}

% Configure page margins with geometry
\geometry{left=1.4cm, top=.8cm, right=1.4cm, bottom=1.8cm, footskip=.5cm}
\fontdir[fonts/]
\colorlet{awesome}{awesome-red}

% Set false if you don't want to highlight section with awesome color
\setbool{acvSectionColorHighlight}{true}

% If you would like to change the social information separator from a pipe (|) to something else
\renewcommand{\acvHeaderSocialSep}{\quad\textbar\quad}

%-------------------------------------------------------------------------------
%       CJK (Chinese/Japanese/Korean) Font Support
%-------------------------------------------------------------------------------
% Add Chinese font support using xeCJK
\usepackage{xeCJK}
% Set Chinese fonts (fallback to available system fonts)
\setCJKmainfont{Droid Sans Fallback}
\setCJKsansfont{Droid Sans Fallback}
\setCJKmonofont{Droid Sans Fallback}

%-------------------------------------------------------------------------------
%	PERSONAL INFORMATION
%	Comment any of the lines below if they are not required
%-------------------------------------------------------------------------------
% Available options: circle|rectangle,edge/noedge,left/right
\name{叶**}{-}
\position{Python}

%-------------------------------------------------------------------------------
\begin{document}

% Print the header with above personal informations
% Give optional argument to change alignment(C: center, L: left, R: right)
\makecvheader[C]

% Print the footer with 3 arguments(<left>, <center>, <right>)
% Leave any of these blank if they are not needed
\makecvfooter
  {\today}
  {叶** -~~~\textperiodcentered~~~Résumé}
  {\thepage}

%-------------------------------------------------------------------------------
%	CV/RESUME CONTENT
%	Each section is imported separately, open each file in turn to modify content
%-------------------------------------------------------------------------------
\cvsection{Summary}
\begin{cvitems}  \item {大模型应用开发}  \item {期望职位: Python @ 南京, 50-80K}  \item {期望职位: 其他后端开发 @ 南京, 50-80K}  \item {期望职位: Java @ 南京, 50-80K}\end{cvitems}


\cvsection{Skills}
\begin{cvskills}  \cvskill
    {Professional Skills}
    {编程语言:Python,Jaya,Rust(了解)。,技术框架:LangChain、LangGraph、Spring、Akka。,系统架构:LLM Multi-Aaent构建、多线程与异步编程、分布式任务调度。}\end{cvskills}


\cvsection{Professional Experience}
\begin{cventries}  \cventry
    {Java}
  {北京字节跳动科技有限公司}
    {None}
    {2025.01 - 至今}
    {      \begin{cvitems}        \item {大模型应用开发}      \end{cvitems}    }  \cventry
    {大模型算法}
  {字节跳动}
    {None}
    {2024.01 - 2025.04}
    {      \begin{cvitems}        \item {负责多个ai驱动的数据分析平台的核心模块开发与系统设计,涵盖数据接入、调度执行、归因分析等场景,DataWeaver Canvas}        \item {一款ai辅助的画布式数据分析工具,支持多源数据导入、一键JOIN、剪贴簿采集及自然语言分析等能力}        \item {画布:一个可以无限延展的数据工作空间,可以将几乎任意数据以卡片的形式导入或粘贴到画布中,卡片之间的连线则表示数据的上下游关联。画布还支持便签、图片、飞书文档等元素多数据源:DataWeaver支持风神数据源,同时支持Excel、CSV等常用数据格式。DataWeaver支持一键刷新数据源,保持数据的新鲜度。}        \item {剪贴簿:可以将风神链接、Excel等文件直接通过剪贴簿复制进画布,也可以在任意网页、桌面应用中将所选择的数据通过剪贴簿引入画布。}        \item {一键Join:无论你是想将两张表拼合成一张大宽表,或是想一次性查询多个ID的明细表,都可以通过JOIN一键完成}        \item {ai智能辅助:通过自然语言,可以实现复杂的高级数据分析功能。}        \item {工作内容:}        \item {负责数据源模块的后端开发,设计并实现了高性能的数据接入链}        \item {路;采用Apache Arrow的Parauet格式进行数据压缩与序列化,结合Rust的Polars库实现Excel等数据格式的解析与转换,大幅提升了大体量数据的导入效率,参与ai智能分析模块的服务端设计与实现,支持用户通过自然语言查询并获取数据分析结果,显著降低了非技术用户的使用门槛,实现剪贴簿功能的后端逻辑,支持用户使用异构数据源快速采集并结构化粘贴数据,提升数据整理与分析效率}        \item {DataWeaver Metric Report是一个面向团队负责人、一线运营与数据分析师,解决团队目标追踪困难、分析协作低效、数据分散等问题,构建自动化、可沉淀的数据指标报告系统。}        \item {支持以可视化方式快速生成指标报告,1分钟内完成配置并定时运行。提供历史记录、周报复制、仪表盘/数据门户同步等一站式数据流}        \item {转能力。}        \item {报告更新时自动飞书推送异常预警与初步归因结论,提升运营响应}        \item {速度。}        \item {提供分析画布与归因记录机制,沉淀方法论与分析过程,支持持续优化。}        \item {降低接入成本,仅需导入风神链接即可生成指标大幅减少分析师重复工作,提高跨角色协作效率}        \item {工作内容:}        \item {作为后端开发,主导实现了指标导入与配置功能,支持用户通过导入风神可视化链接快速生成可定时运行的指标报告,整体接入流程}        \item {控制在1分钟内完成。}        \item {设计并实现了定时调度系统,支持自定义周期拉取与刷新数据,保障数据报告的时效性与稳定性}        \item {开发了基于飞书开放平台的推送机器人,支持异常预警、初步归因分析及分析画布内容的自动推送,提升团队响应效率}        \item {实现了归因结果记录机制,用户可在群内通过回复机器人交互方式}        \item {记录异常分析结论,构建可追溯的指标分析链路}        \item {项目大量复用了前期DataWeaver Canvas项目的基础设施与通用能力,体现了系统设计的前瞻性与良好的架构可复用性,帮助团队在两周内高效完成了初版研发与上线DataTalks}        \item {DataTalks是字节跳动数据平台团队推出的一款ai数据助手,支持用户通过自然语言对话进行数据查询与分析。系统基于Multi-Agent架构,将数据分析任务拆分为多个专职Agent,并由中央控制器}        \item {CopilotAgent进行任务分配与调度,提升整体系统的灵活性与可扩展性。}        \item {CopilotAgent:系统入口,基于标准ReAct思路构建,负责全局任务编排与流水线调度:}        \item {DataQueryAgent:接收用户查询请求,基于中立结构化查询语言}        \item {(Semantic Layer)构建查询逻辑,避免直接生成SQL带来的性能和兼容性问题,实现Token消耗显著减少、生成速度提升}        \item {可将结构化查询语言转换风神等可视化数据产品的查询语义}        \item {提升用户理解与修改查询条件的效率}        \item {通过MetricLayer支持多数据源(ClickHouse、Hive 等)接入,降低不同存储引擎之间SQI转换带来的不稳定性,提升了系统兼容性;}        \item {AdvancedDataAnalysisAgent:对查询结果进行二次分析,基于CodeAct思路构建,负责生成Python代码对数据进行处理RootCauseAgent:负责用户数据归因分析的处理。}        \item {工作内容:}        \item {查询条件推荐能力的实现;从语义化查询语句中提取查询条件组合,按用户、模块、数据集维度进行聚合统计,构建查询行为画像;}        \item {将高频查询组合写入向量数据库,实现基于数据集上下文的查询条件推荐,提升用户查询效率与体验。}        \item {ai Insight}        \item {ai Insight是一款面向业务分析场景,支持基于单指标的探索式分析的产品。用户可通过类似思维导图(Mindmap)的交互方式自由构建分析路径,系统提供多维度的数据归因能力,助力深入洞察业务问题}        \item {工作内容:}        \item {算法调度层的开发,接收用户在前端配置的分析路径,根据不同算法所需动态聚合数据并路由至对应归因算法模块,在调度环节引入Java 21 的虚拟线程(FiberThread),大幅提升并发处理能力}        \item {有效缩短系统响应时间}        \item {设计并落地基于领域驱动设计(DDD)的模块架构,将用户节点配}        \item {置(Node)、算法执行器(NodeExecutor)与SQL查询器}        \item {(SalExecutor)进行抽象解耦,显著提升算法模块的扩展性,使得新算法平均可在一天内完成接入并上线}      \end{cvitems}    }  \cventry
    {Java}
  {阿里巴巴}
    {None}
    {2017.06 - 2024.01}
    {      \begin{cvitems}        \item {负责客服人资成本管控系统的设计与研发}        \item {负责客服人资领域通用计算引擎的设计与研发,}      \end{cvitems}    }\end{cventries}


\cvsection{Projects}
\begin{cventries}  \cventry
    {Java}
  {客服绩效薪资计算引擎}
    {None}
    {2021.03 - 至今}
    {      \begin{cvitems}        \item {支持复杂业务规则配置、跨源数据查询及高并发计算的通用计算平台。}        \item {支持基于表达式的规则计算与自动依赖解析,提升规则配置的灵活性;}        \item {支持多数据源接入(如数据仓库、API等),适配不同业务系统的指标查询需求;}        \item {提供定制化计算逻辑的扩展机制,如SLA组件、排名函数等,满足复杂业务场景;}        \item {引擎具备良好的水平扩展能力,支撑高并发计算任务}        \item {支持业务数据隔离,保障多业务线并行使用时的数据安全与隔离}        \item {性。}        \item {工作内容}        \item {配置管理模块;设计并实现规则配置存储系统,支持表达式、指标及业务定制函数(如SLA、排名)的统一配置管理,确保多业务数据隔离与权限控制}        \item {依赖解析器:基于拓扑排序算法解析业务规则公式,将规则拆解为可执行的有向无环图(DAG),用于任务依赖调度运行时调度系统:构建任务调度与执行框架,支持根据任务规模动}        \item {态选择本地多线程或基于Akka的分布式并发执行方案,兼顾性能与扩展性;}      \end{cvitems}    }\end{cventries}


\cvsection{Education}
\begin{cventries}  \cventry
    {本科}
  {贵州大学}
    {None}
    {2008 - 2012}
    {      \begin{cvitems}        \item {专业: 网络工程}      \end{cvitems}    }\end{cventries}


%-------------------------------------------------------------------------------
\end{document}

